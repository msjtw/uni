\documentclass[11pt]{article}

\usepackage[T1]{fontenc}
\usepackage[polish]{babel}

% Margins
\topmargin=-0.45in
\evensidemargin=0in
\oddsidemargin=0in
\textwidth=6.5in
\textheight=9.0in
\headsep=0.25in

\title{%
    Rozwiązywanie rówania nieliniowego \\
    metodą Newtona Raphsona.}
\author{ Stanisław Fiedler }
\date{\today}

\begin{document}
\maketitle	

\section{Zastosowanie}
    Funkcja \verb|newton_raphson_i| znajduje przedział zawierający miejsce zerowe równania z zadaną dokładnością.
\section{Opis metody}
    Kolejne przybliżenia wartości miejsca zerowego fukcji za pomocą metody Newtona Raphsona wyznaczane są wzorem:
    $$x_{i+1} = x_i - \frac{f'(x_i)+\sqrt[]{[f'(x_i)]^2-2f(x_i)f''(x_i)}}{f''(x_i)}, i = 0, 1, ...$$
    gdzie $x_0$ jest dana. 
    Proces kończy się kiedy szerokość przedziału zawierającego dwa ostatnie przybilżenia jest miejszy od zadanego na początku $\varepsilon$.
\section{Wywołanie funkcji}
    \verb|newton_raphson_i(res, f, df, d2f ,mit, eps, fatx, it);|
\section{Dane}
    \begin{description}
        \item[x] - początkowe przybliżenie miejsca zerowego.
        \item[f] - wskaźnik do funkcji zwracającej wartość $f(x)$.
        \item[df] - wskaźnik do funkcji zwracającej wartość $f'(x)$.
        \item[d2f] - wskaźnik do funkcji zwracającej wartość $f''(x)$.
        \item[mit] - maksymalna liczba iteracji.
        \item[eps] - błąd wyzaczanego miejsca zerowego.
    \end{description}
\section{Wynik}
    \begin{description}
        \item[x] - wyznaczone przybliżenie miejsca zerowego.
        \item[fatx] - wartość funkcji dla wyznaczonego miejsca zerowego.
        \item[it] - ilość wykonanych iteracji.
    \end{description}
\section{Inne parametry}
    Funkcja \verb|newton_raphson_i| zwraca status wykonia. 
    Poszczególne wartości oznaczają:
    \begin{itemize}
        \item 1 - mit < 1.
        \item 2 - dla pewnego x: $f''(x) = 0$.
        \item 3 - w mit iteracjach nie osiągnięto dokładości eps.
        \item 4 - dla pewnego x: $[f'(x)]^2 -2f(x)f''(x) < 0$
        \item 0 - w innych przypadkach.
    \end{itemize}
\section{Typy parametrów}
    \begin{description}
        \item[Interval]: x, fatx; 
        \item[Interval (*)(Interval)]: f, df, d2f;
        \item[int]: mit, it;
        \item[\_Float128]: eps;
    \end{description}
\section{Identyfikatory nielokalne}
    \begin{description}
        \item[Interval] - klasa wprowadzająca arytmetykę przedziałową.
        \item[Interval (*)(Interval)] - fukcja przyjmująca jako jedyny argument i zwracająca obiekt klasy Interval.
    \end{description}
\pagebreak
\section{Tekst funkcji}
    \begin{verbatim}
        int newton_raphson_i(Interval &x, Interval (*f)(Interval), 
                    Interval (*df)(Interval), Interval (*d2f)(Interval),
                    const int mit, const _Float128 eps, Interval &fatx, int &it){
            int st;
            Interval dfatx, d2fatx, p, v, w, xh, x1, x2, ret;

            if(mit < 1){
                return 1;
            }
            st = 3;
            it = 0;
            do{
                it++;
                fatx = f(x);
                dfatx = df(x);
                d2fatx = d2f(x);
                p = dfatx*dfatx - 2*fatx*d2fatx;
                if(con_zero(d2fatx)){
                    st = 2;
                }
                else if(neg(p)){
                    st = 4;
                }
                else{
                    xh = x;
                    w = abs(xh);
                    p = sqrt(p);
                    x = x-(dfatx+p)/d2fatx;
                    ret = make_intr(x, xh);
                    if( ret.width() < eps ){
                        st = 0;
                    }
                }
            } while(it < mit and st == 3);
            if(st == 0 or st == 3){
                x = ret;
                fatx = f(x);
            }
            return st;
        } 
    \end{verbatim}

\pagebreak
\section{Przykłady}
    \begin{enumerate}
        \item Równanie
        $$f(x) = x^3 - 3x -1$$
        Dane:\\
        \begin{tabular}{r l}
            \hline
            x:&-0.5\\
            mit:&10\\
            eps:&1e-16\\
            \hline
        \end{tabular}\\\\
        Wyniki:\\
        \begin{tabular}{r l}
            \hline
            x:&-3.47296355333860697703433253538629587230160763599116e-01\\
            fx:&0.0e+00\\
            it:&4\\
            st:&0\\
            \hline
        \end{tabular}
        \item Równanie
        $$f(x) = x^3 - 3x -1$$
        Dane:\\
        \begin{tabular}{r l}
            \hline
            x:&[-0.5, -0.5]\\
            mit:&10\\
            eps:&1e-16\\
            \hline
        \end{tabular}\\\\
        Wyniki:\\
        \begin{tabular}{r l}
            \hline
            x:&[ -3.47296355333860697703433253782741063236497078931272e-01, \\
              &-3.47296355333860697703433253538612783491395984966298e-01 ]\\
            szerokość:&2.44128279745101093964974138431842897246009738490558e-28\\
            fx:&[ -8.83808288514330339349461749863251854013475577315218e-29, \\
              &7.32340639143079594832094780601746367656068766038846e-28 ]\\
            szerokość:&8.20721467994512628767040955588071553057416323770368e-28\\
            it:&4\\
            st:&0\\
            \hline
        \end{tabular}
        \item Równanie
        $$f(x) = x^3 - 3x -1$$
        Dane:\\
        \begin{tabular}{r l}
            \hline
            x:&[-10, -10]\\
            mit:&10\\
            eps:&1e-16\\
            \hline
        \end{tabular}\\\\
        Wyniki:\\
        \begin{tabular}{r l}
            \hline
            st:&4\\
            \hline
        \end{tabular}\\
        dla pewnego x: $[f'(x)]^2 -2f(x)f''(x) < 0$
    \end{enumerate}
\pagebreak
\tableofcontents
\end{document}
