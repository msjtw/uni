\documentclass[a4paper, 11pt]{article}
\usepackage[polish]{babel}
\usepackage[T1]{fontenc}
\usepackage{hyperref}
\usepackage{array}
\usepackage{amssymb}
\hypersetup{
    colorlinks,
    citecolor=black,
    filecolor=black,
    linkcolor=black,
    urlcolor=black
}
\usepackage{graphicx}

\usepackage{tikz}
\usetikzlibrary{fit,arrows,matrix,positioning, calc, shapes.gates.logic.IEC, shapes.gates.logic.US}
\tikzstyle{branch}=[fill,shape=circle,minimum size=3pt,inner sep=0pt]


\title{%
	\vspace{-3.5cm}
       \large Sprawozdanie Laboratorium Mikroelektronika \\
       \huge Obsługa programu LTSpice}

\author{Stanisław Fiedler 160250}
\date{LAB 1, 15 października 2024}

\begin{document}

\maketitle
\tableofcontents

\section{Na otrzymanych wynikach symulacji zaznaczyć obszary liniowy oraz nasycenia tranzystora
nMOS. W oparciu o wiedzę z podstaw elektroniki podać i omówić stosowne wzory wyjaśniające zasadę
działania tranzystora nMOS.}\label{sec:zadanie_} % (fold)

% section Zadanie1 (end)


\end{document}

