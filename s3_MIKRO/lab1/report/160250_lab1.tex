\documentclass[a4paper, 11pt]{article}
\usepackage[polish]{babel}
\usepackage[T1]{fontenc}
\usepackage{hyperref}
\usepackage{array}
\usepackage{amssymb}
\hypersetup{
    colorlinks,
    citecolor=black,
    filecolor=black,
    linkcolor=black,
    urlcolor=black
}
\usepackage{graphicx}

\usepackage{tikz}
\usetikzlibrary{fit,arrows,matrix,positioning, calc, shapes.gates.logic.IEC, shapes.gates.logic.US}
\tikzstyle{branch}=[fill,shape=circle,minimum size=3pt,inner sep=0pt]


\title{%
       \large Sprawozdanie Laboratorium Mikroelektronika \\
       \huge Obsługa programu LTSpice}

\author{Stanisław Fiedler 160250}
\date{LAB 1, 15 października 2024}

\begin{document}

\maketitle
\tableofcontents

\section{Zadanie 1}

\subsection{Wyjaśnić czym różni się analiza Transient od analizy stałoprądowej DC.}\label{sub:1.1} % (fold)


% subsection 1.1 (end)

\section{Zadanie 2}

\subsection{Przedstawić wyniki symulacyjne oraz wyjaśnić zasadę działania obwodu.}\label{sub:2.1} % (fold)

% subsection 2.1 (end)

\subsection{aproponować dowolną zmianę w obwodzie. Opisać wprowadzoną zmianę oraz przedstawić dla
niej wyniki symulacyjne.}\label{sub:2.2} % (fold)

% subsection 2.2 (end)

\end{document}
